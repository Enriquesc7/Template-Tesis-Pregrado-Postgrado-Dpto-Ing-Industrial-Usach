\chapter*{CAPITULO IV TITULO DEL CAPITULO}
\addcontentsline{toc}{chapter}{CAPITULO IV TITULO DEL CAPITULO}
%-------------------------------------------------------------------------
En este capítulo se pretende profundizar en las citas, por lo que se presentará una gran variedad de ejemplos para hacerlos dependiendo del tipo de cita deseado.

% Titulo de ejemplo 4.1
\section{4.1 Referencias en APA}

\subsection{4.1.1 Cita como mención del/los autor(es)}

% Cita en mención de un autor
\subsection*{Mencionando un autor}

Contenido de ejemplo según \textcite{articulo:ejemplo_2}, que explica las diferentes formas de citar en Latex.


%Cita en mención de varios autores
\subsection*{Mencionando varios autores}

Contenido de ejemplo según \textcite{libro:ejemplo_varios_autores}, que explica las diferentes formas de citar en Latex.


\subsection{4.1.2 Citando al final del parrafo o idea}

\subsection*{Citando un autor}

Lorem ipsum dolor sit amet, consectetur adipiscing elit, sed do eiusmod tempor incididunt ut labore et dolore magna aliqua. Ut enim ad minim veniam, quis nostrud exercitation ullamco laboris nisi ut aliquip ex ea commodo consequat \parencite{articulo:ejemplo_2}.


\subsection*{Citando varios autores}

Lorem ipsum dolor sit amet, consectetur adipiscing elit, sed do eiusmod tempor incididunt ut labore et dolore magna aliqua. Ut enim ad minim veniam, quis nostrud exercitation ullamco laboris nisi ut aliquip ex ea commodo consequat \parencite{libro:ejemplo_varios_autores}.


\subsection*{Citando a más de 3 autores}
En este caso se presentará cuando se quiere referir al autor a mitad del texto, por ejemplo cuando se dice: según \textcite{articulo:ejemplo_1} dice que \textit{"Podremos citar de muchas maneras diferentes en Latex, y podemos obtener más información en la documentación de Overleaf: https://es.overleaf.com/learn"} o también cuando queremos hacer la cita al final del texto como en este caso \parencite{articulo:ejemplo_1}.


\subsection{4.1.3 Citando con número de pagina}

Para citar con numeros de pagina puede utilizar los mismos comandos explicados en los subcapitulos anteriores, por ejemplo, según \textcite[p. 7]{libro:ejemplo_2}, que explica las diferentes formas de citar en Latex.

Si queremos citar con un intevalo de paginas podemos hacerlo con solo escribir el intervalo antre guion, como lo menciona \textcite[p. 32-39]{articulo:ejemplo_1}. También podemos hacerlo con 'parencite', como se ve a continuación \parencite[p. 32-39]{articulo:ejemplo_1}.

En el caso que los números de páginas que se desea referenciar no son correlativos, sino diferentes, se hace tan solo separandolos con una coma, como se ve a continuación \parencite[p. 23, 37, 39]{articulo:ejemplo_1}.

Agregamos algunas citas para tenerlas de ejemplo, ya que, según \textcite{Gorman2020} es una buena forma de comprender mejor las cosas. Al final de este parrafo agregaremos una ultima cita para terminar \parencite{Garcia2019}
%-------------------------------------------------------------------------